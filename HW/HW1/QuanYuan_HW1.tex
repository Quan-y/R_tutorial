\documentclass[]{article}
\usepackage{lmodern}
\usepackage{amssymb,amsmath}
\usepackage{ifxetex,ifluatex}
\usepackage{fixltx2e} % provides \textsubscript
\ifnum 0\ifxetex 1\fi\ifluatex 1\fi=0 % if pdftex
  \usepackage[T1]{fontenc}
  \usepackage[utf8]{inputenc}
\else % if luatex or xelatex
  \ifxetex
    \usepackage{mathspec}
  \else
    \usepackage{fontspec}
  \fi
  \defaultfontfeatures{Ligatures=TeX,Scale=MatchLowercase}
\fi
% use upquote if available, for straight quotes in verbatim environments
\IfFileExists{upquote.sty}{\usepackage{upquote}}{}
% use microtype if available
\IfFileExists{microtype.sty}{%
\usepackage{microtype}
\UseMicrotypeSet[protrusion]{basicmath} % disable protrusion for tt fonts
}{}
\usepackage[margin=1in]{geometry}
\usepackage{hyperref}
\hypersetup{unicode=true,
            pdfborder={0 0 0},
            breaklinks=true}
\urlstyle{same}  % don't use monospace font for urls
\usepackage{color}
\usepackage{fancyvrb}
\newcommand{\VerbBar}{|}
\newcommand{\VERB}{\Verb[commandchars=\\\{\}]}
\DefineVerbatimEnvironment{Highlighting}{Verbatim}{commandchars=\\\{\}}
% Add ',fontsize=\small' for more characters per line
\usepackage{framed}
\definecolor{shadecolor}{RGB}{248,248,248}
\newenvironment{Shaded}{\begin{snugshade}}{\end{snugshade}}
\newcommand{\KeywordTok}[1]{\textcolor[rgb]{0.13,0.29,0.53}{\textbf{#1}}}
\newcommand{\DataTypeTok}[1]{\textcolor[rgb]{0.13,0.29,0.53}{#1}}
\newcommand{\DecValTok}[1]{\textcolor[rgb]{0.00,0.00,0.81}{#1}}
\newcommand{\BaseNTok}[1]{\textcolor[rgb]{0.00,0.00,0.81}{#1}}
\newcommand{\FloatTok}[1]{\textcolor[rgb]{0.00,0.00,0.81}{#1}}
\newcommand{\ConstantTok}[1]{\textcolor[rgb]{0.00,0.00,0.00}{#1}}
\newcommand{\CharTok}[1]{\textcolor[rgb]{0.31,0.60,0.02}{#1}}
\newcommand{\SpecialCharTok}[1]{\textcolor[rgb]{0.00,0.00,0.00}{#1}}
\newcommand{\StringTok}[1]{\textcolor[rgb]{0.31,0.60,0.02}{#1}}
\newcommand{\VerbatimStringTok}[1]{\textcolor[rgb]{0.31,0.60,0.02}{#1}}
\newcommand{\SpecialStringTok}[1]{\textcolor[rgb]{0.31,0.60,0.02}{#1}}
\newcommand{\ImportTok}[1]{#1}
\newcommand{\CommentTok}[1]{\textcolor[rgb]{0.56,0.35,0.01}{\textit{#1}}}
\newcommand{\DocumentationTok}[1]{\textcolor[rgb]{0.56,0.35,0.01}{\textbf{\textit{#1}}}}
\newcommand{\AnnotationTok}[1]{\textcolor[rgb]{0.56,0.35,0.01}{\textbf{\textit{#1}}}}
\newcommand{\CommentVarTok}[1]{\textcolor[rgb]{0.56,0.35,0.01}{\textbf{\textit{#1}}}}
\newcommand{\OtherTok}[1]{\textcolor[rgb]{0.56,0.35,0.01}{#1}}
\newcommand{\FunctionTok}[1]{\textcolor[rgb]{0.00,0.00,0.00}{#1}}
\newcommand{\VariableTok}[1]{\textcolor[rgb]{0.00,0.00,0.00}{#1}}
\newcommand{\ControlFlowTok}[1]{\textcolor[rgb]{0.13,0.29,0.53}{\textbf{#1}}}
\newcommand{\OperatorTok}[1]{\textcolor[rgb]{0.81,0.36,0.00}{\textbf{#1}}}
\newcommand{\BuiltInTok}[1]{#1}
\newcommand{\ExtensionTok}[1]{#1}
\newcommand{\PreprocessorTok}[1]{\textcolor[rgb]{0.56,0.35,0.01}{\textit{#1}}}
\newcommand{\AttributeTok}[1]{\textcolor[rgb]{0.77,0.63,0.00}{#1}}
\newcommand{\RegionMarkerTok}[1]{#1}
\newcommand{\InformationTok}[1]{\textcolor[rgb]{0.56,0.35,0.01}{\textbf{\textit{#1}}}}
\newcommand{\WarningTok}[1]{\textcolor[rgb]{0.56,0.35,0.01}{\textbf{\textit{#1}}}}
\newcommand{\AlertTok}[1]{\textcolor[rgb]{0.94,0.16,0.16}{#1}}
\newcommand{\ErrorTok}[1]{\textcolor[rgb]{0.64,0.00,0.00}{\textbf{#1}}}
\newcommand{\NormalTok}[1]{#1}
\usepackage{graphicx,grffile}
\makeatletter
\def\maxwidth{\ifdim\Gin@nat@width>\linewidth\linewidth\else\Gin@nat@width\fi}
\def\maxheight{\ifdim\Gin@nat@height>\textheight\textheight\else\Gin@nat@height\fi}
\makeatother
% Scale images if necessary, so that they will not overflow the page
% margins by default, and it is still possible to overwrite the defaults
% using explicit options in \includegraphics[width, height, ...]{}
\setkeys{Gin}{width=\maxwidth,height=\maxheight,keepaspectratio}
\IfFileExists{parskip.sty}{%
\usepackage{parskip}
}{% else
\setlength{\parindent}{0pt}
\setlength{\parskip}{6pt plus 2pt minus 1pt}
}
\setlength{\emergencystretch}{3em}  % prevent overfull lines
\providecommand{\tightlist}{%
  \setlength{\itemsep}{0pt}\setlength{\parskip}{0pt}}
\setcounter{secnumdepth}{0}
% Redefines (sub)paragraphs to behave more like sections
\ifx\paragraph\undefined\else
\let\oldparagraph\paragraph
\renewcommand{\paragraph}[1]{\oldparagraph{#1}\mbox{}}
\fi
\ifx\subparagraph\undefined\else
\let\oldsubparagraph\subparagraph
\renewcommand{\subparagraph}[1]{\oldsubparagraph{#1}\mbox{}}
\fi

%%% Use protect on footnotes to avoid problems with footnotes in titles
\let\rmarkdownfootnote\footnote%
\def\footnote{\protect\rmarkdownfootnote}

%%% Change title format to be more compact
\usepackage{titling}

% Create subtitle command for use in maketitle
\newcommand{\subtitle}[1]{
  \posttitle{
    \begin{center}\large#1\end{center}
    }
}

\setlength{\droptitle}{-2em}

  \title{}
    \pretitle{\vspace{\droptitle}}
  \posttitle{}
    \author{}
    \preauthor{}\postauthor{}
    \date{}
    \predate{}\postdate{}
  

\begin{document}

\section{Homewoke 1}\label{homewoke-1}

\textbf{UNI:} qy2205\\
\textbf{Name:} Quan Yuan\\
\textbf{Email:}
\href{mailto:quan.yuan@columbia.edu}{\nolinkurl{quan.yuan@columbia.edu}}

\subsubsection{Part I: Importing Data into
R}\label{part-i-importing-data-into-r}

\begin{Shaded}
\begin{Highlighting}[]
\CommentTok{# (1) import data}
\NormalTok{titanic <-}\StringTok{ }\KeywordTok{read.table}\NormalTok{(}\StringTok{'Titanic.txt'}\NormalTok{, }\DataTypeTok{as.is =} \OtherTok{TRUE}\NormalTok{, }\DataTypeTok{header =} \OtherTok{TRUE}\NormalTok{)}

\CommentTok{# (2) columns and rows}
\KeywordTok{cat}\NormalTok{(}\StringTok{"The number of data columns is "}\NormalTok{, }\KeywordTok{ncol}\NormalTok{(titanic))}
\end{Highlighting}
\end{Shaded}

\begin{verbatim}
## The number of data columns is  12
\end{verbatim}

\begin{Shaded}
\begin{Highlighting}[]
\KeywordTok{cat}\NormalTok{(}\StringTok{"The number of data rows is "}\NormalTok{, }\KeywordTok{nrow}\NormalTok{(titanic))}
\end{Highlighting}
\end{Shaded}

\begin{verbatim}
## The number of data rows is  891
\end{verbatim}

\begin{Shaded}
\begin{Highlighting}[]
\CommentTok{# (3) creat new variable called survived.word}
\NormalTok{survive_word =}\StringTok{ }\KeywordTok{ifelse}\NormalTok{(titanic[}\StringTok{'Survived'}\NormalTok{] }\OperatorTok{==}\StringTok{ }\DecValTok{0}\NormalTok{, }\StringTok{'dead'}\NormalTok{, }\StringTok{'survived'}\NormalTok{)}
\NormalTok{titanic[}\StringTok{"Survived.Word"}\NormalTok{] <-}\StringTok{ }\NormalTok{survive_word}
\CommentTok{# head(titanic, 1)}
\end{Highlighting}
\end{Shaded}

\subsubsection{Part II: Exploring the Data in
R}\label{part-ii-exploring-the-data-in-r}

\begin{Shaded}
\begin{Highlighting}[]
\CommentTok{# (1) calculate the mean in survived, age, fare columns}
\KeywordTok{apply}\NormalTok{(titanic[,}\KeywordTok{c}\NormalTok{(}\StringTok{'Survived'}\NormalTok{, }\StringTok{'Age'}\NormalTok{, }\StringTok{'Fare'}\NormalTok{)], }\DecValTok{2}\NormalTok{, mean)}
\end{Highlighting}
\end{Shaded}

\begin{verbatim}
##   Survived        Age       Fare 
##  0.3838384         NA 32.2042080
\end{verbatim}

\begin{Shaded}
\begin{Highlighting}[]
\KeywordTok{cat}\NormalTok{(}\StringTok{"The mean of survived is 0.383, which means the survival rate is very low. Only 3-4 survived among 10 people. The mean of Age is NA becasue Age column exists NA."}\NormalTok{)}
\end{Highlighting}
\end{Shaded}

\begin{verbatim}
## The mean of survived is 0.383, which means the survival rate is very low. Only 3-4 survived among 10 people. The mean of Age is NA becasue Age column exists NA.
\end{verbatim}

\begin{Shaded}
\begin{Highlighting}[]
\CommentTok{# (2) Compute the proportion of female passengers who survived the titanic disaster}
\NormalTok{fsurvive =}\StringTok{ }\KeywordTok{factor}\NormalTok{(titanic[,}\StringTok{'Survived'}\NormalTok{])}
\NormalTok{fsex =}\StringTok{ }\KeywordTok{factor}\NormalTok{(titanic[,}\StringTok{'Sex'}\NormalTok{])}
\NormalTok{summary_}\DecValTok{2}\NormalTok{ =}\StringTok{ }\KeywordTok{summary}\NormalTok{(}\KeywordTok{split}\NormalTok{(titanic[,}\StringTok{'PassengerId'}\NormalTok{], }\KeywordTok{list}\NormalTok{(fsurvive, fsex)))}
\KeywordTok{cat}\NormalTok{(summary_}\DecValTok{2}\NormalTok{)}
\end{Highlighting}
\end{Shaded}

\begin{verbatim}
##  81 233 468 109 -none- -none- -none- -none- numeric numeric numeric numeric
\end{verbatim}

\begin{Shaded}
\begin{Highlighting}[]
\NormalTok{prob1 =}\StringTok{ }\KeywordTok{as.numeric}\NormalTok{(summary_}\DecValTok{2}\NormalTok{[}\DecValTok{2}\NormalTok{])}\OperatorTok{/}\KeywordTok{nrow}\NormalTok{(titanic)}
\NormalTok{prob1 =}\StringTok{ }\KeywordTok{round}\NormalTok{(prob1, }\DecValTok{2}\NormalTok{)}
\KeywordTok{cat}\NormalTok{(}\StringTok{"the proportion of female passengers who survived the titanic disaster is "}\NormalTok{, prob1)}
\end{Highlighting}
\end{Shaded}

\begin{verbatim}
## the proportion of female passengers who survived the titanic disaster is  0.26
\end{verbatim}

\begin{Shaded}
\begin{Highlighting}[]
\CommentTok{# (3) Of the survivors, compute the proportion of female passengers.}
\NormalTok{prob2 =}\StringTok{ }\KeywordTok{as.numeric}\NormalTok{(summary_}\DecValTok{2}\NormalTok{[}\DecValTok{2}\NormalTok{])}\OperatorTok{/}\NormalTok{(}\KeywordTok{as.numeric}\NormalTok{(summary_}\DecValTok{2}\NormalTok{[}\DecValTok{2}\NormalTok{]) }\OperatorTok{+}\StringTok{ }\KeywordTok{as.numeric}\NormalTok{(summary_}\DecValTok{2}\NormalTok{[}\DecValTok{4}\NormalTok{]))}
\NormalTok{prob2 =}\StringTok{ }\KeywordTok{round}\NormalTok{(prob2, }\DecValTok{2}\NormalTok{)}
\KeywordTok{cat}\NormalTok{(}\StringTok{"Of the survivors, the proportion of female passengers is "}\NormalTok{, prob2)}
\end{Highlighting}
\end{Shaded}

\begin{verbatim}
## Of the survivors, the proportion of female passengers is  0.68
\end{verbatim}

\begin{Shaded}
\begin{Highlighting}[]
\CommentTok{# (4) }
\end{Highlighting}
\end{Shaded}


\end{document}
